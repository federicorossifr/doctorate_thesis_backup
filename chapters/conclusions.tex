\chapter{Conclusions}
In this work, we presented an in-depth overview of the different aspects of real number representations, from the IEEE standard to the novel and innovative alternatives. In particular, we analysed the posit format, reporting the main advantages and disadvantages as well as its core properties. Moreover, we showed novel advancements in the posit format with particular optimized implementations of non-linear functions; we managed to implement fast and approximated versions of non-linear activation functions that offer a significative speed-up with little accuracy degradation.

Secondly, we presented our implementation of a posit library with its high-level application programming interface as well as its integration with common machine learning frameworks such as Tensorflow. On this topic, we also showed our results on posit accuracy when employed in different deep neural network tasks; we demonstrated how 16-bit posits can replace IEEE binary32 numbers in a series of DNN tasks without significant loss in accuracy. Furthermore, we proved how 8-bit posits can be a decent trade-off between accuracy and memory footprint when used in such tasks.

Finally, we moved to the hardware side with two main topics: (i) a lightweight posit processing unit for data compression integrated into a RISC-V core and (ii) a pipelined full posit processing unit configurable core that also supports algebraic operations between posits. For the first topic, we developed an implementation flow that goes from a new instruction definition to the implementation in the RISC-V simulator and, in the end, to the integration in a RISC-V core. For the second topic, we implemented a configurable posit processing unit that supports any posit configuration up to 64-bit words and all algebraic operations. Moreover, we combined literature solutions to the complex division task to deliver an efficient, yet accurate, solution to the problem.

To conclude, some ongoing activities may constitute future development on the topics presented in this work:
\begin{itemize}
    \item Extension of the Full Posit Processing Unit to support fused-multiply-add operation via an exact accumulator following the Posit Standard Group specification\footnote{\url{https://posithub.org/docs/posit_standard-2.pdf}, pp. 6}
    \item Dynamic power characterization of the full posit processing unit (in collaboration with Politecnico di Milano): detailed estimation of dynamic power consumption of the component based on real-world task switching activity
    \item Integration of the full posit processing unit inside the STX accelerator of the next EPI processor (in collaboration with Fraunhofer Fraunhofer Institut and Instituto Superior Técnico of the Universidade de Lisboa).
\end{itemize}